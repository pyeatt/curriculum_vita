\documentclass[11pt]{article}
\usepackage{times}
\pagestyle{empty}
\textwidth 6.5in
\textheight 9in
\evensidemargin 0in
\oddsidemargin 0in
\topmargin 0in
\headheight 0in
\headsep 0in

\begin{document}

\section*{Career goals}

% What attracts the candidate to a career in academia.

\centerline{\large\bf Teaching Interests}
\vspace{\baselineskip}

I find teaching to be very personally rewarding.  I have taught two
courses, and am currently teaching two more.  While this is not a
great deal of experience, it has prompted me to consider what
subjects I am most competent to teach.  Unsurprisingly, the subjects
that I enjoy teaching are the subjects that I know the most about.
My strengths are artificial intelligence, digital logic, computer
architecture, and operating systems.  In the course of developing my
dissertation, I have also learned a great deal about linear
programming.

\subsubsection*{Self Evaluation}

The first course that I taught was the freshman level introduction
to computer science.  It was as much a learning experience for me as
it was for my students.  I did a good job of teaching the course,
considering it was my first course.  By the end of the semester,
I was experimenting to improve my teaching style.


My second class was a 200 level course in digital logic.  I had an
excellent teacher in this course, and decided to model my teaching
style on his.  It worked very well.  I was able to engage my
students and draw their interest.  The student evaluations for this
course were overwhelmingly positive.  In looking over them, I noted
one student who said, ``Mr. Pyeatt did a very good job of trying to
make a very boring subject interesting.''  This was the most
negative comment I received, and at the same time, the highest
praise.

Currently I am teaching a 300 level course in operating systems.  I
enjoy teaching this course, but have noticed that my students lack
experience in assembly language and do not have a good grasp of the
hardware.  It is difficult to explain interrupt service routines,
for instance, to someone who does not really understand subroutines,
CPU registers, or the stack. I am overcoming this frustration by
focusing on the most important topics and explaining them in depth,
with supporting material from outside the textbook.  My teaching
style continues to improve, and my lectures are much more
interactive.  The results of the first exam are in, and the grades
fall on a perfect bell curve.

I am also teaching a course in programming languages.  Languages are
not my greatest strength.  Preparing lectures for this course takes
far more time than the other courses I have taught.  I do not handle
questions as well in this course as I can in other courses, and
student interaction is not as high as I would like.  I am putting
more effort into this course in order to improve class involvement
and my presentation of the material.

My MS thesis advisor, Brian Oldham at Texas Tech, and my PhD
advisor, Adele Howe at Colorado State, are both good role models for
directing graduate students. I have observed them treating each
student differently, according to the needs and capabilities of the
student.  I believe that I will be able to perform the function of a
graduate advisor competently.  I enjoy teaching one on one, although
I can be very demanding.  I strive for excellence in teaching and
ask that my students perform to the best of their abilities.

\subsubsection*{Goals and Areas of Teaching Expertise}

My immediate goals are to further improve my teaching techniques and
increase my breadth of knowledge so that I can competently teach a
broader range of courses.  Currently, I can teach the following
courses well.


\begin{description}
\item[Introductory courses:]  
Introduction to Computer Science and
programming, data structures and algorithms, digital logic, 
artificial intelligence, compiler construction,
computer graphics, and operating systems.

\item[Advanced courses:]  
Computer organization and design, operating systems,
computer graphics, artificial intelligence, computer
architecture.

\item[Graduate courses:]
Introduction to artificial intelligence, reinforcement learning,
neural networks, genetic algorithms, and planning with completely
and partially observable Markov decision processes.

\end{description}
\end{document}
