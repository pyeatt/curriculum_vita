\documentclass[11pt]{resume}
\usepackage{ifthen}
\usepackage{times}
\usepackage{fancyheadings}

\textwidth 7.25in
\evensidemargin -0.5in
\oddsidemargin -0.5in
\topmargin -0.25in
\textheight 9.5in

\leftcolwidth{1in}
\colsep{0.1in}
%\outdent{0.25in}
\outdent{0in}
 
\headheight 3\normalbaselineskip
\headsep \normalbaselineskip
\addtolength{\topmargin}{-3\normalbaselineskip}
\addtolength{\textheight}{-1\normalbaselineskip}
\lhead{}
\rhead{\thepage}
\chead{Larry D. Pyeatt}
\cfoot{}
\lfoot{}
\rfoot{}

\begin{document}
\name{\begin{minipage}{\textwidth}\vspace*{-\baselineskip}\begin{center}Larry
      D. Pyeatt\end{center}\end{minipage}}

\pagestyle{fancyplain}
\thispagestyle{empty}

\address{Texas Tech University\\ Computer Science Department \\ Lubbock,
  Texas 79409 \\ (806) 742--3527\\ pyeatt@cs.ttu.edu } \line


\section{Role on Proposed Investigation}
{}
{Design and implementation of hardware and software for intelligent
control and data acquisition.}

\section{Education}
{Doctor of Philosophy in Computer Science}
{\\Colorado State University, 1999. \\
  Dissertation: {\it Integration of Partially Observable Markov Decision
    Processes and Reinforcement Learning for Simulated Robot
    Navigation}\\
  Committee: Adele Howe (Chair), Charles Anderson, Darrell Whitley, Wade
  Troxell }
   
\section{} 
{Master of Science in Computer Science}
{\\Texas Tech University, 1991. \\
  Thesis: {\it Application of the Neural Ring Pattern Classifier to Speech
    Recognition}}

   
\section{} 
{Bachelor of Science in Computer Science} {\\Texas Tech University, 1988.}

\section{Research Interests}  
{} 
{Partially Observable Markov Decision Processes, Reinforcement Learning,
  Function Approximation, Robotics and Agent Architectures, Real-time and
  Embedded Systems }

\section{Professional Experience}
{Assistant Professor} {Texas Tech University, Lubbock, 9/99--present.
  Currently focusing on research in Robotics, Partially Observable Markov
  Decision Processes, and Reinforcement Learning.  Have taught courses in
  robotics, digital logic, operating systems, reinforcement learning, and
  artificial intelligence.  }

\section{}
{Graduate Research Assistant/Doctoral Candidate}{Colorado State University,
  Fort Collins, 9/93--9/99. Explored several areas of AI before focusing on
  reinforcement learning and partially observable Markov decision processes
  in the context of robotics. Developed a three-level robot control
  architecture using a symbolic planner at the highest level, POMDP planning
  at the middle level, and reinforcement learning at the low level.
  Implemented middle and low levels and demonstrated that a robot could
  learn low level actions as necessary to complete its task and could modify
  its low level actions to adapt to sensor and effector failures.  }

\section{}
{Senior Information Systems Programmer} {Texaco Inc., Houston, Texas,
  9/91--9/93. Applied Artificial Intelligence techniques to Texaco business
  needs.}
   
\section{}{Embedded Control Systems Design Engineer} {Applied
  Hydraulics, Lubbock, Texas, 1/87--7/88. Designed and built microprocessor
  and sequential logic based systems for industrial control and data
  acquisition.  Designed circuit boards for the systems.  Supervised
  technical staff who were responsible for constructing the hardware
  designs.  Designed and wrote software for the new systems in assembly and
  C.  Supervised for all stages of the process: assessing customer needs,
  creating design specifications, implementing the systems, installing at
  the customer's site, and providing post-delivery technical assistance.}


\newcommand{\bibfile}{complete}

\begin{citations}{Related Publications}{\bibfile}
\nocite{Moore:Quasny:Pyeatt:Sinzinger:C01b}
\nocite{Pyeatt:Howe:J00}
\nocite{Pyeatt:Howe:C01a}
\nocite{Pyeatt:Howe:C99a}
\nocite{Pyeatt:Howe:C98b}
\end{citations}

\begin{citations}{Other Publications}{\bibfile}
\nocite{Pyeatt:Howe:C99c}
\nocite{Howe:Pyeatt:C96}
\nocite{Gruau:Whitley:Pyeatt:C96}
\nocite{Whitley:Mathias:Pyeatt:C95}
\nocite{Whitley:Gruau:Pyeatt:C95}
\end{citations}

\end{document}

