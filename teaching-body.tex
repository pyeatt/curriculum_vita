
I take teaching very seriously and strive to do the best job that I can.  I
have worked in industry and routinely bring that experience to the
classroom to help prepare students to enter the work-force. Several students
have attributed their success in industry to taking one or more of my
classes.  My overall philosophy of teaching can be described as follows:

\begin{description}

  \item[Continuous Improvement:] Not only is Computer Science a rapidly
  changing field, but new  pedagogies are being developed all of
  the time. If a course does not change, then it becomes outdated. This is
  true of any field, but especially true for Computer Science. Thus, I work
  for continuous improvement in my course materials, content, and teaching
  style.  This applies to not only the classes that I am teaching, but also
  to my duty to help determine and improve the curriculum for the
  department.

  \item[Active Learning:] Active learning gets the student involved so that
  they learn the material at a deeper level than rote memorisation.
  In-class discussions and course projects are excellent ways to involve
  the students.  The homework and projects should be chosen carefully to
  reinforce the most important concepts in the course.  As students
  progress and mature, they should take more of the responsibility for
  learning. At some point, they can become their own teachers.  That
  is the point at which they are truly educated.

  \item[Appropriate Rewards:] Students should get the grade they earn.
  What many students want is to get the highest possible grades for the
  least amount of work. That is natural and should be expected.  However,
  teachers have a responsibility to display fairness and integrity.  It is
  important to set expectations, tell the students what the expectations
  are, and tie grades to how well the students meet those expectations.

\end{description}

Courses that I enjoy teaching include robotics, artificial intelligence,
digital logic, architecture, operating systems, assembly language, and
real-time systems.   Courses that I would like to teach if
given the opportunity include genetic algorithms,
speech recognition, planning, machine vision, 
Markov decision processes, discrete mathematics, data structures, system
administration, and compiler construction.  In addition to these
preferences, I am competent and willing to teach any traditional computer
science course at either the undergraduate or graduate level.


My greatest teaching achievement involves a student who had a low GPA and
was in danger of dropping out of the program. He confided in me that his
dream was to be a mission controller for NASA.  He also indicated that he
was interested in robotics, so I told him that I would work with him on two
conditions: he was to make a 4.0 GPA in the coming semester, and meet with
me weekly for an independent study in reinforcement learning.  At the end of
the semester, he had achieved all A's and had a good basic understanding of
reinforcement learning.  More than that, our relationship had developed into
a mentorship. By Fall of his senior year, he was doing research. He
published his first paper as a senior and published another in his first
year of graduate school. Not only did he blossomed academically, but he also
decided to work towards a PhD.  

About one year into his dissertation, we were working on a research project
with a group from NASA and he got the opportunity of a lifetime.  He was
offered a position as a mission controller on the International Space
Station.  I was sad to see him take the position, but also happy for him.
Not many people get to achieve their dreams. My mentorship of him has given
me a new perspective on teaching and advising: Some students need a teacher
to get them interested and involved, and I can be that teacher.  Nothing
could be more personally rewarding.





%%% Local Variables: 
%%% mode: latex
%%% TeX-master: "career_goals"
%%% End: 
