

\section{Professional Experience}
{Assistant Professor} {Texas Tech University, Lubbock, 9/99--present.
  Served as graduate advisor and was responsible for re-structuring and
  improving the Master's and Doctoral programs.  Currently focusing on
  research in Robotics, Partially Observable Markov Decision Processes, and
  Reinforcement Learning.  Taught several graduate and undergraduate
  courses including courses in digital logic, operating systems,
  reinforcement learning, and artificial intelligence.  Received very high
  student evaluations without sacrificing high student performance
  expectations.  Directed research work of several MS students
  and one undergraduate.}

\section{}
{Graduate Research Assistant/Doctoral Candidate}{Colorado State University,
  Fort Collins, 9/93--9/99. Explored several areas of AI before focusing on
  reinforcement learning and partially observable Markov decision processes
  in the context of robotics for my dissertation. Taught courses in digital
  logic and assembly language, operating systems, and programming
  languages.}
% The following sections detail some of the research that 
%  I conducted at Colorado State.}
%      \begin{description}
%      \item[Partially Observable Markov Decision
%        Processes](9/97--9/99)
%        \ Implemented several POMDP solution
%        algorithms. 
%        Extended  the Incremental Pruning algorithm and code to perform
%        distributed execution on several UNIX machines.
%      \item[Robotics] (9/97--9/99)
%        \ Developed a three-level control architecture for the Khepera robot
%        using a symbolic planner at the highest level, POMDP planning at
%        the middle level, and reinforcement learning
%        at the low level.  Implemented middle and low levels and
%        demonstrated that a robot could learn low level
%        extended actions
%        as necessary to complete its task and could modify its low level
%        actions to adapt to sensor  and effector failures.
%      \item[Neural Networks and Reinforcement Learning](9/96--9/97)
%        \ Experimented with reinforcement
%        learning using neural networks for function
%        approximation.  Ran experiments using a Pente simulator, pole-balancing
%        simulator, and the Robot Auto Racing Simulator (RARS).  Added a 
%        high-level controller to the RARS system to allow switching between
%        learned behaviors.  Invented a new function approximation
%        technique based on decision trees.  Showed that it was superior to
%        the neural network approach.
%      \item[Structure in Discrete Event Sequences](9/95--9/97)
%        \ Developed method to detect dependencies in program execution traces
%        and construct a Semi-Markov model of the underlying system behavior.
%        Worked with Adele Howe, Eric Dahlman
%        and Gabriel Somlo to write CLASPWeb which allows anyone with a web
%        browser to use the code. %(http://satchmo.cs.colostate.edu:4936)
%      \item[Geographical Information Systems (GIS)](5/95--8/95)
%        \ Developed method to normalize lighting and remove shadows 
%        from aerial and satellite terrain photographs.  The software
%        worked by texture mapping the photograph onto the corresponding
%        USGS elevation data, calculating the position of the sun at the
%        time the photograph was taken, and applying an inverse lighting
%        calculation.  The normalized photograph could then be texture mapped
%        onto the USGS elevation data and a forward lighting model could
%        be applied to show lighting conditions at any time of day or year.
%      \item[Genetic Algorithms](9/93--5/95)
%        \ Developed system to 
%        track the ranking of hyperplanes during genetic search;
%        compared dynamic ranking with static ranking.
%        Wrote code to evolve neural network weights. Compared the
%        results with Fr\'ed\'eric Gruau's Cellular Encoding method.  
%%      \item[Other projects]
%%        Completed several smaller projects, including a ray tracer,
%%        a computer vision system, and an optimizing compiler for a language
%%        similar to C.
%\end{description}


\section{}
    {Senior Information Systems Programmer} {Texaco Inc., Houston,
    Texas, 9/91--9/93.  Applied  Artificial
    Intelligence techniques to Texaco business needs.}
%  Delivered applications on PC, Unix, and IBM
%    mainframe platforms.  Performed system administration for Unix
%    environment.  Used a wide variety of systems, languages, tools,
%    and techniques.}
% Major projects included:}
%\begin{description}
%   \item[SpectraMan](9/92--9/94) \ Ultraviolet Fluorescence Spectra
%   Manager. The software was used to automate the process of
%   identifying and characterizing unknown hydrocarbon mixtures.  It
%   kept a database of thousands of known 3D ultraviolet fluorescence
%   spectra.  The chemist could use a neural network, embedded in the
%   software, to match an unknown sample to every other
%   sample in the database.  The neural net selected a user-specified
%   number of similar spectra that the chemist could then view and
%   print, along with additional information in the database, in
%   order to make a final determination about the characteristics of
%   the unknown sample.  This software greatly reduced the time
%   required to perform an analysis and enabled a large increase in
%   the productivity of the chemists.  I wrote the prototype in
%   assembly and C on a PC running DOS.  The prototype showed what
%   was possible, but was deemed too slow and could only allow
%   viewing of one spectrum at a time.  The final version of the
%   software was written in C using the X Window System, Motif,
%   OpenGL, and an SQL database server.  It was developed on Sun and
%   delivered on SGI.  The lead chemist, Dr. Marilyn Reyes, and I
%   were awarded United States Patent 5424959 for this software.

%   \item[Reservoir Modeling Expert System](11/9--11/92) \ The reservoir management
%   team at Texaco gathers production data on every reservoir that
%   Texaco owns.  Once they have sufficient data on a particular
%   reservoir, they try to fit the data to one of several
%   mathematical models.  If they select the correct model, they can
%   accurately predict the production of the reservoir and select an
%   appropriate production schedule.  There were very few people
%   experienced in selecting the model, and several experts had recently
%   retired or were preparing to retire. The team leader was
%   faced with having an inexperienced staff and no one to
%   train them.  We built an expert system
%   for selecting the reservoir model, allowing the team to continue
%   its work with a reduced and less experienced staff.
%\end{description}
    
 \section{}
  {Graduate Research Assistant}
   {Texas Tech University, Lubbock, Texas,  8/88--9/91. Research
  in neural networks and speech recognition.}
%\begin{description}
%      \item[Speech Recognition] Developed speaker independent speech
%      recognition system for digits.  The speech recognition system
%      followed the now standard model of signal preprocessing, fast
%      Fourier transform, Mel scale extraction, segmentation using a
%      hidden Markov model, and recognition.  The recognition section
%      used a neural ring pattern classifier.  
%      \item[Embedded Control] Evaluated hardware and
%      operating systems for embedded distributed control of silicon
%      wafer processing machinery.  Made recommendation to the
%      company that funded the study.  Taught Introduction to
%      Computer Science course using Pascal.  Taught digital
%      electronics lab.  Supervised construction of circuits by the
%      students.  Assisted students in the implementation of their
%      circuit designs.  \end{description}

%\section{}{Self-Employed}{8/88--7/89. 
%      Developed accounting and subscriber list software for small newspapers. 
%      Wrote code in C and Pascal on IBM PC and Apple Macintosh systems.}
   
\section{}{Embedded Control Systems Design Engineer} {Applied
  Hydraulics, Lubbock, Texas, 1/87--7/88.  Designed and built
  microprocessor and sequential logic based systems for industrial control
  and data acquisition. }
% Designed circuit boards for the systems.
%    Supervised technical staff who were responsible for constructing
%    the hardware designs.  Designed and wrote software for the new
%    systems in assembly and C.  Supervised for all stages of
%    the process: assessing customer needs, creating design
%    specifications, implementing the systems, installing at the
%    customer's site, and providing post-delivery technical
%    assistance.}

