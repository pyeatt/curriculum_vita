\documentclass[12pt,letter]{resume}
\usepackage{ifthen}
\usepackage{times}
\usepackage{fancyhdr}

%% \textwidth 7in
%% \evensidemargin -0.175in
%% \oddsidemargin -0.175in
\textwidth 7in
\evensidemargin -0.25in
\oddsidemargin -0.25in
\topmargin -0.25in
\textheight 9.5in

\newboolean{prospectus} 
\setboolean{prospectus}{false}

\leftcolwidth{1in}
\colsep{0.1in}
\outdent{0.25in}
\outdent{0in}
 
\headheight 3\normalbaselineskip
\headsep \normalbaselineskip
\addtolength{\topmargin}{-3\normalbaselineskip}
\addtolength{\textheight}{-1\normalbaselineskip}
\lhead{}
\rhead{\thepage}
\chead{Larry D. Pyeatt, PhD\\Curriculum Vitae}
\cfoot{}
\lfoot{}
\rfoot{}

\newcommand\myaddress{13286 Edgewood Place\\Piedmont, SD 57769 \\
         ldpyeatt@gmail.com \\ (806)241-6151}

\renewcommand\myaddress{Department of Computer Science and Electrical Engineering\\
  South Dakota School of Mines and Technology\\
         larry.pyeatt@sdsmt.edu}

\begin{document}
\name{\begin{minipage}{\textwidth}\vspace*{-\baselineskip}\begin{center}Larry D. Pyeatt, PhD \\ Curriculum Vitae\end{center}\end{minipage}}

\pagestyle{fancyplain}
\thispagestyle{empty}

\address{\myaddress}
\line

\section{Education}
  {Doctor of Philosophy in Computer Science}
  {\\Colorado State University, 1999 \\
  Dissertation:  {\it Integration of Partially Observable Markov
Decision Processes and Reinforcement Learning for Simulated Robot 
Navigation}\\ 
  Committee: Adele Howe (Chair), Charles Anderson, 
  Darrell Whitley, Wade Troxell }
   
\section{} 
  {Master of Science in Computer Science}    
  {\\Texas Tech University, 1991 \\
  Thesis:   {\it Application of the Neural Ring Pattern Classifier
  to Speech Recognition}\\
  Committee: W. J. Bryan Oldham (Chair), Thomas M. English, Donald Gustafson}
   
\section{} 
  {Bachelor of Science in Computer Science}    
  {\\Texas Tech University, 1988 \\
  Minor in Psychology with additional course work in Mathematics and 
  Electrical Engineering}

\section{Research Interests}  
  {} {Machine Learning, Neural Networks, Partially Observable Markov
    Decision Processes, Reinforcement Learning, Function
    Approximation, Bioinformatics, Agent Architectures, Real-time and
    Embedded Systems, Robotics }

\section{Honors and Awards}  
{Academic}   
{\\
Upsilon Pi Epsilon (computer science honor society), Texas Tech University,
1989\\
Third Place Team, ACM International Programming Competition,
Louisville, Kentucky,  1989\\
%President's List, Texas Tech, Spring 1986 \\
%Dean's List, Texas Tech, Spring 1986 \\
Engineering Academic Scholarship, Texas Tech University, 1983}

\section{}
{Texaco}   
{\\Patent Letter, August 30, 1995\\
Patent Application Award, July 19, 1993 \\
Exploration \& Production Technology Department Award for 
Outstanding Supplier, February 23, 1993  \\
Individual Outstanding Contribution (IOC) Award for Innovation,
August 5, 1992}

%\section{Languages}
%{}{English, Russian}

\section{Professional Experience}
{Associate Professor} {South Dakota School of Mines and Technology,
Rapid City, South Dakota, 8/2012--present.
Research in Robotics, Partially Observable Markov Decision Processes, and
Reinforcement Learning.  Teach several graduate and undergraduate courses,
in areas of Robotics, AI, Computer Forensics, and core Computer Science
curriculum.  Direct research work of graduate
students.}

\section{}
{Associate Professor} {Texas Tech University, Lubbock, 1/2006--7/2012.
Research in Robotics, Partially Observable Markov Decision Processes, and
Reinforcement Learning.  Teach several graduate and undergraduate courses,
in areas of Robotics, AI, Computer Forensics, and core Computer Science
curriculum.  Direct research work of graduate
students.}

\section{}
{Associate Department Chair}{Texas Tech University at Abilene, 8/2007 --
8/2010.  All duties of Associate professor, plus manage the Computer
Science department at Abilene.
Developed strategic plan to increase enrollment of graduate students,
improve Faculty recruitment and retention, and increase research productivity
and external funding.}

\section{}
{Visiting Associate Professor} {University of Missouri, Rolla,
1/2005--12/2005. Taught graduate course in Markov Decision Processes and
graduate course in Reinforcement Learning.  Conducted research on two
research projects in collaboration with Donald Wunsch.  The projects
involved optimal routing in disruption tolerant networks and threat detection
and evaluation using smart sensors.   }

\section{}
{Assistant Professor} {Texas Tech University, Lubbock,
9/99--12/2004.  
Performed research in Robotics, Partially
Observable Markov Decision Processes, and Reinforcement Learning.  Taught
graduate and undergraduate courses including courses in digital logic,
operating systems, reinforcement learning, and artificial intelligence.
Directed research work of MS and PhD students.
As graduate advisor, led efforts to restructure and improve
the graduate programs, resulting in enrollment growth for
the MS and PhD programs.}

\section{}
{Lecturer}{Colorado State University,
  Fort Collins, 9/98--9/99. Taught courses in digital
  logic and assembly language, operating systems, and programming
  languages.}

\section{}
{Graduate Research Assistant}{Colorado State University, Fort Collins,
  multiple appointments, 9/93--9/98.  Research in areas of partially
  observable Markov decision processes, robotics, neural networks and
  reinforcement learning, finding structure in discrete event sequences,
  geographical information systems (GIS), and genetic algorithms.}
%  in the context of robotics for my dissertation. 
% The following sections detail some of the research that 
%  I conducted at Colorado State.}
%      \begin{description}
%      \item[Partially Observable Markov Decision
%        Processes](9/97--9/99)
%        \ Implemented several POMDP solution
%        algorithms. 
%        Extended  the Incremental Pruning algorithm and code to perform
%        distributed execution on several UNIX machines.
%      \item[Robotics] (9/97--9/99)
%        \ Developed a three-level control architecture for the Khepera robot
%        using a symbolic planner at the highest level, POMDP planning at
%        the middle level, and reinforcement learning
%        at the low level.  Implemented middle and low levels and
%        demonstrated that a robot could learn low level
%        extended actions
%        as necessary to complete its task and could modify its low level
%        actions to adapt to sensor  and effector failures.
%      \item[Neural Networks and Reinforcement Learning](9/96--9/97)
%        \ Experimented with reinforcement
%        learning using neural networks for function
%        approximation.  Ran experiments using a Pente simulator, pole-balancing
%        simulator, and the Robot Auto Racing Simulator (RARS).  Added a 
%        high-level controller to the RARS system to allow switching between
%        learned behaviors.  Invented a new function approximation
%        technique based on decision trees.  Showed that it was superior to
%        the neural network approach.
%      \item[Structure in Discrete Event Sequences](9/95--9/97)
%        \ Developed method to detect dependencies in program execution traces
%        and construct a Semi-Markov model of the underlying system behavior.
%        Worked with Adele Howe, Eric Dahlman
%        and Gabriel Somlo to write CLASPWeb which allows anyone with a web
%        browser to use the code. %(http://satchmo.cs.colostate.edu:4936)
%      \item[Geographical Information Systems (GIS)](5/95--8/95)
%        \ Developed method to normalize lighting and remove shadows 
%        from aerial and satellite terrain photographs.  The software
%        worked by texture mapping the photograph onto the corresponding
%        USGS elevation data, calculating the position of the sun at the
%        time the photograph was taken, and applying an inverse lighting
%        calculation.  The normalized photograph could then be texture mapped
%        onto the USGS elevation data and a forward lighting model could
%        be applied to show lighting conditions at any time of day or year.
%      \item[Genetic Algorithms](9/93--5/95)
%        \ Developed system to 
%        track the ranking of hyperplanes during genetic search;
%        compared dynamic ranking with static ranking.
%        Wrote code to evolve neural network weights. Compared the
%        results with Fr\'ed\'eric Gruau's Cellular Encoding method.  
%%      \item[Other projects]
%%        Completed several smaller projects, including a ray tracer,
%%        a computer vision system, and an optimizing compiler for a language
%%        similar to C.
%\end{description}


\section{}
    {Senior Information Systems Programmer} {Texaco Inc., Houston 
      9/91--9/93.  Applied  Artificial
    Intelligence techniques to Texaco business needs.}
%  Delivered applications on PC, Unix, and IBM
%    mainframe platforms.  Performed system administration for Unix
%    environment.  Used a wide variety of systems, languages, tools,
%    and techniques.}
% Major projects included:}
%\begin{description}
%   \item[SpectraMan](9/92--9/94) \ Ultraviolet Fluorescence Spectra
%   Manager. The software was used to automate the process of
%   identifying and characterizing unknown hydrocarbon mixtures.  It
%   kept a database of thousands of known 3D ultraviolet fluorescence
%   spectra.  The chemist could use a neural network, embedded in the
%   software, to match an unknown sample to every other
%   sample in the database.  The neural net selected a user-specified
%   number of similar spectra that the chemist could then view and
%   print, along with additional information in the database, in
%   order to make a final determination about the characteristics of
%   the unknown sample.  This software greatly reduced the time
%   required to perform an analysis and enabled a large increase in
%   the productivity of the chemists.  I wrote the prototype in
%   assembly and C on a PC running DOS.  The prototype showed what
%   was possible, but was deemed too slow and could only allow
%   viewing of one spectrum at a time.  The final version of the
%   software was written in C using the X Window System, Motif,
%   OpenGL, and an SQL database server.  It was developed on Sun and
%   delivered on SGI.  The lead chemist, Dr. Marilyn Reyes, and I
%   were awarded United States Patent 5424959 for this software.

%   \item[Reservoir Modeling Expert System](11/9--11/92) \ The reservoir management
%   team at Texaco gathers production data on every reservoir that
%   Texaco owns.  Once they have sufficient data on a particular
%   reservoir, they try to fit the data to one of several
%   mathematical models.  If they select the correct model, they can
%   accurately predict the production of the reservoir and select an
%   appropriate production schedule.  There were very few people
%   experienced in selecting the model, and several experts had recently
%   retired or were preparing to retire. The team leader was
%   faced with having an inexperienced staff and no one to
%   train them.  We built an expert system
%   for selecting the reservoir model, allowing the team to continue
%   its work with a reduced and less experienced staff.
%\end{description}
    
 \section{}
  {Graduate Research Assistant}
   {Texas Tech University, Lubbock,  8/88--9/91. Research
  in neural networks and speech recognition.}
%\begin{description}
%      \item[Speech Recognition] Developed speaker independent speech
%      recognition system for digits.  The speech recognition system
%      followed the now standard model of signal preprocessing, fast
%      Fourier transform, Mel scale extraction, segmentation using a
%      hidden Markov model, and recognition.  The recognition section
%      used a neural ring pattern classifier.  
%      \item[Embedded Control] Evaluated hardware and
%      operating systems for embedded distributed control of silicon
%      wafer processing machinery.  Made recommendation to the
%      company that funded the study.  Taught Introduction to
%      Computer Science course using Pascal.  Taught digital
%      electronics lab.  Supervised construction of circuits by the
%      students.  Assisted students in the implementation of their
%      circuit designs.  \end{description}

%\section{}{Self-Employed}{8/88--7/89. 
%      Developed accounting and subscriber list software for small newspapers. 
%      Wrote code in C and Pascal on IBM PC and Apple Macintosh systems.}
   
\section{}{Embedded Control Systems Engineer} {Applied
  Hydraulics, Lubbock, Texas, 1/87--7/88.  Designed and built
  microprocessor and sequential logic based systems for industrial control
  and data acquisition. }
% Designed circuit boards for the systems.
%    Supervised technical staff who were responsible for constructing
%    the hardware designs.  Designed and wrote software for the new
%    systems in assembly and C.  Supervised for all stages of
%    the process: assessing customer needs, creating design
%    specifications, implementing the systems, installing at the
%    customer's site, and providing post-delivery technical
%    assistance.}

%\pagebreak

\newcommand{\bibfile}{$HOME/bibliography/LarryPubs}


%\begin{citations}{Accepted Papers Awaiting Publication\vspace*{-\baselineskip}}{\bibfile}
%
%\end{citations}
\begin{citations}{Textbooks}{$HOME/bibliography/LarryPubs}
\nocite{textbook2025}
\nocite{textbook2020}
\nocite{textbook2016}
\end{citations}

\begin{citations}{Refereed Journal Articles}{$HOME/bibliography/LarryPubs}
\nocite{Moore:Pyeatt:2014}
\nocite{Moore:Doufas:Pyeatt:2011a}
\nocite{Pyeatt:Howe:J00}
\end{citations}


\begin{citations}{Refereed Conference Papers}{$HOME/bibliography/LarryPubs}
\nocite{Caudle:Pyeatt:Fleming:Hoover:2019}
  \nocite{Randrianasolo:Pyeatt:2018}
  \nocite{Caudle:Karlsson:Pyeatt:2015a}
\nocite{Caudle:Karlsson:Pyeatt:2014a}
\nocite{Randrianasolo:Pyeatt:2014b}
\nocite{Randrianasolo:Pyeatt:2014a}
\nocite{Caudle:Karlsson:Pyeatt:2014a}
\nocite{Randrianasolo:Pyeatt:2012a}
\nocite{Randrianasolo:Pyeatt:2012b}
\nocite{Borera:Pyeatt:2011}
\nocite{Borera:Moore:Doufas:Pyeat:2011}
\nocite{randrianasolo:pyeatt:2011}
\nocite{shukla:pyeatt:2011}
\nocite{Borera:Randrianasolo:Mahdi:Pyeatt:2010}
\nocite{Randrianasolo:Pyeatt:2010}
\nocite{Moore:Panousis:IAAI:2010}
\nocite{Moore:Pyeatt:Doufas:2009b}
\nocite{Moore:Pyeatt:Doufas:2009}
\nocite{KimPyeattWunsch2009}
\nocite{KimPyeattWunsch2007}
\nocite{Helm:Cooke:Rushton:Pyeatt:Becker:2006}
\nocite{Moore:Quasny:Sinzinger:Pyeatt:C04}
\nocite{Quasny:Pyeatt:C04}
\nocite{Li:PyeattC04a}
\nocite{Li:PyeattC04b}
\nocite{Li:PyeattC04c}
\nocite{Li:PyeattC04d}
\nocite{Quasny:Garcia:Barnes:Pyeatt:C04}
\nocite{pyeatt2003a}
\nocite{quasny2003a}
\nocite{Moore:Quasny:Pyeatt:Sinzinger:C01b}
\nocite{Pyeatt:Howe:C01a}
\nocite{Pyeatt:Howe:C99a}
\nocite{Pyeatt:Howe:C99c}
\nocite{Pyeatt:Howe:C98b}
\nocite{Pyeatt:Howe:C98a}
\nocite{Howe:Pyeatt:C96}
\nocite{Gruau:Whitley:Pyeatt:C96}
\nocite{Whitley:Mathias:Pyeatt:C95}
\nocite{Whitley:Gruau:Pyeatt:C95}
\nocite{Pyeatt:Oldham:C91}
\end{citations}


%% Really should move this paper here
%% \begin{citations}{Unrefereed Conference Papers}{$HOME/bibliography/LarryPubs}
%% \nocite{Caudle:Karlsson:Pyeatt:2014a}
%% \end{citations}


%\begin{citations}{Papers in Preparation}{\bibfile}
%\nocite{Pyeatt:Quasny:C03}
%\nocite{Moore:Quasny:Pyeatt:C03}
%\end{citations}

\begin{citations}{Doctoral \\ Dissertation}{\bibfile}
\nocite{PyeattTHESIS}
\end{citations}

\begin{citations}{Master's \\ Thesis}{\bibfile}
\nocite{Pyeatt:91}
\end{citations}

\begin{citations}{Unrefereed Symposia \& Workshops}{$HOME/bibliography/LarryUnref}
\nocite{Pyeatt:Howe:WS99a}
\nocite{Pyeatt:Howe:FS98}
\end{citations}


\begin{citations}{Unrefereed Technical Reports}{\bibfile}
\nocite{Stevens:Pyeatt:Houlton:Goss:95}
\nocite{Pyeatt:Howe:TR98c}
\end{citations}

\begin{citations}{Professional
Presentations}{$HOME/bibliography/LarryInvited}
\nocite{Present14}
\nocite{Present13}
\nocite{Present11}
\nocite{Present03a}
\nocite{Present03b}
\nocite{Present02a}
\nocite{Present01b}
\nocite{Present01a}
\nocite{Present00a}
\nocite{Present99c}
\nocite{Present99b}
\nocite{Present99a}
\nocite{Present98c}
\nocite{Present98b}
\nocite{Present98a}
\nocite{Present97}
\nocite{Present96}
\nocite{Present91}
\end{citations}

\begin{citations}{Patent}{$HOME/bibliography/LarryPatent}
\nocite{Reyes:Pyeatt95}
\end{citations}

\newcommand{\grantfile}{$HOME/bibliography/LarryGrants}


%\begin{citations}{Pending Proposals}{\grantfile}
%\end{citations}


\begin{citations}{Grants Received}{\grantfile}
  \nocite{Grant2018A}
  \nocite{Grant2014A}
  \nocite{Grant2013A}
  \nocite{GrantH}
  \nocite{GrantG}
  \nocite{GrantA}
  \nocite{GrantB}
  \nocite{GrantC}
  \nocite{GrantD}
  \nocite{GrantE}
  \nocite{GrantF}
\end{citations}


\begin{citations}{Recent Proposals}{\grantfile}
  \nocite{GrantProposal2018A}
  \nocite{GrantProposal2017A}
  \nocite{GrantProposal2016A}
  \nocite{GrantProposal2016B}
  \nocite{WhitePaper2015}
  \nocite{GrantProposal2015A}
  \nocite{GrantProposal2014C}
  %  \nocite{GrantProposal2014B}
  \nocite{GrantProposal2014A}
  \nocite{WhitePaper2013}
  \nocite{GrantProposal2013B}
  \nocite{GrantProposal2013A}
\end{citations}

\begin{citations}{Other Proposals}{\grantfile}
  \nocite{GrantProposalO}
  \nocite{GrantProposalN}
  \nocite{GrantProposalM}
  \nocite{GrantProposalL}
  \nocite{GrantProposalK}
  \nocite{GrantProposalJ}
  \nocite{GrantProposalI}
  \nocite{GrantProposalH}
  \nocite{GrantProposalG}
  \nocite{GrantProposalF}
  \nocite{GrantProposalE}
  \nocite{GrantProposalD}
  \nocite{GrantProposalC}
  \nocite{GrantProposalB}
\end{citations}


%Progress}{\bibfile}
%\nocite{PyeattTHESIS}
%\nocite{PyeattWIPa}
%\nocite{PyeattWIPb}
%\nocite{PyeattWIPc}
%\nocite{PyeattWIPd}
%\end{citations}


\section{Courses Taught}{Undergraduate}
        {\\
          \hspace*{1em} Foundations of Electrical and Computer Engineering (EE 120)\\
          \hspace*{1em} Introduction Digital Systems (CENG 142)\\
          \hspace*{1em} Digital Systems (CENG 242)\\          
          \hspace*{1em} Microprocessor Design (CENG 442)\\
          \hspace*{1em} Real-time Operating Systems (CENG 448/548)\\
          \hspace*{1em} Operating Systems (CSC/CENG 458)\\
          \hspace*{1em} Introduction to Robotics (CSC 415/515)\\
          \hspace*{1em} Assembly Language (CSC/CENG 320)\\
          \hspace*{1em} Computer Science I (CSC 150)\\
          \hspace*{1em} Computer Organization and Architecture (CSC 317)\\
          \hspace*{1em} Data Structures (CSC 315)\\
          \hspace*{1em} Senior Design (CSC 465)\\
          \hspace*{1em} Cooperative Education (Cp 297/397/497/697)\\
\hspace*{1em} Introduction to Systems Programming\\
\hspace*{1em} Introduction to Digital Logic\\
\hspace*{1em} Advanced Digital Projects\\
\hspace*{1em} Introduction to AI Robotics\\
\hspace*{1em} Programming Languages
}

\section{}{Graduate}
{\\
  \hspace*{1em} Introduction to Robotics (CSC 415/515)\\
\hspace*{1em} Digital Forensics\\
\hspace*{1em} Computer Architecture\\
\hspace*{1em} Markov Decision Processes\\
\hspace*{1em} Reinforcement Learning\\
\hspace*{1em} Advanced Operating Systems\\
\hspace*{1em} Intelligent Systems\\
\hspace*{1em} Introduction to AI Robotics\\
\hspace*{1em} All-terrain Robotics
}


\section{Graduate Students Advised}{Master's Students}
        {\\
          Ashley Schnetzer, Graduated, May, 2025\\
          David Matthews, Graduated, December, 2024\\
          Devon Scheider, Graduated, December, 2023\\
  Kyle MacMillan, Graduated, December 2020\\
  Andrew Stelter, Graduated, December 2020\\
  Derek Stotz, Graduated, May 2016\\
  Tetsuya Idota, Graduated May, 2015\\
  Amit Yadav, Graduated May, 2012\\
  Shubham Shukla, graduated in May, 2011\\
  Mahdi Naser-Moghadasi, graduated in May, 2010\\
  Arisoa Randrianasolo, graduated in May, 2010\\
            }

        
  \section{}{}{Eddy Borera, graduated in  May, 2010\\
Nguyen Bach, graduated in  May, 2010\\
  Derik Dalton, graduated December, 2009\\
  Roger Coffey, graduated May, 2009\\
  ChengCheng Li, graduated in May, 2005\\
  Karan Gupta, graduated in May, 2005\\
  Krishnan Pazhayanoor,graduated in December, 2004\\
  Julian Hooker, graduated in May, 2004\\
  Todd Quasny, graduated in December, 2003\\
  Bharani Ellore, graduated in December, 2002\\
  Srividya Kona, graduated in May, 2002\\
  Ajay Bansal, graduated in May, 2002\\
}

\section{}{Doctoral Students}
{
  \\Eddy Borera, expected Graduated December 2012\\
  Arisoa Randrianasolo, Graduated May 2012\\
  Brett Moore, Graduated April, 2010\\
 }


\section{Departmental Service}{ABET Coordinator 2023-present.}{Preparing for 2028 ABET review cycle.}

\vspace*{-1.25\baselineskip}
\section{}{Faculty Senator 2013-present.}{}

\vspace*{-1.25\baselineskip}
\section{}{Programming Team Coach 2016-2022}{Ran mock competitions and managed study sessions for the programming teams. Determined which students would be on each team.  Managed registration for the competition and ran the competition site.}


\section{}{Managed Senior Design Team 2015-2016}{We are working on
  coordinated control of an autonomous quad-rotor and an unmanned
  ground vehicle to assist in fighting forest fires.}


\section{}{2013 International Electro/Information Technology
  Conference}
{\\
\hspace*{1em} Worked as Reviewer\\
\hspace*{1em} Served as Session chair\\
}

\vspace*{-1.25\baselineskip}
\section{}{Participated in preparation for ABET accreditation 2013}{}

\vspace*{-1.25\baselineskip}
\section{}{Assisted with West River Math Contest 2012 and 2013}{}

\vspace*{-1.25\baselineskip}
\section{}{CSC and CSR Curriculum Committees 2012--Present}{}

\vspace*{-1.25\baselineskip}
\section{}{Associate department chair, TTU CS at Abilene 2007--2010}
{\\
\hspace*{1em} Lead Abilene branch of the Texas Tech Computer Science Department\\
\hspace*{1em} Developed strategy to increase student enrollment\\
\hspace*{1em} Developed strategy to increase research funding and productivity\\
\hspace*{1em} Developed strategy to recruit and retain top faculty\\
}

\vspace*{-1.25\baselineskip}
\section{}{Department Unix Administrator 1999--2008}
{\\
  \hspace*{1em} Developed System Administrator Guidelines for the CS department network\\
  \hspace*{1em} Developed System Usage Policies for the CS department network\\
\hspace*{1em} Configured servers to provide more reliable service\\
\hspace*{1em} Installed numerous software packages on server and clients\\
\hspace*{1em} Set up accounts for all students enrolled in CS courses\\
\hspace*{1em} Provided email lists for faculty, staff, and students\\
\hspace*{1em} Set up web-based system support request forms\\
\hspace*{1em} Provided systems support to other faculty, staff, and students\\
}

\vspace*{-1.25\baselineskip}
\section{}{Coached Programming Team 1999--2002}{}

\vspace*{-1.25\baselineskip}
\section{}{Graduate Advisor 2000--2001}
{\\
\hspace*{1em} Re-structured the degree requirements for MS degree\\
\hspace*{1em} Created new forms and procedures to improve consistency and enforce requirements\\
\hspace*{1em} Worked to improve consistency in admissions\\
\hspace*{1em} Developed new leveling requirements and created mechanisms to ensure compliance\\
\hspace*{1em} Instituted policies that encourage students to take the thesis option\\
\hspace*{1em} Drove the creation of posters and brochures to recruit graduate students\\
}

\vspace*{-1.25\baselineskip}
\section{}{Organized the UIL Computer Science competition 1999-2005}{}

\vspace*{-1.25\baselineskip}
\section{}{Served on faculty recruiting committees 2001--2003}{}

\vspace*{-1.25\baselineskip}
\section{}{Served on multiple M.S. thesis committees 2000--Present}{}

\section{International Service}{External Examiner}
{\\
PhD defense of Adam Milstein, ``Improved Particle Filter Based Localization and Mapping,''
University of Waterloo, Waterloo, CA, March 5, 2008.
}

\section{\ }{Program Committees}
        {\\
2004 International Conference on Machine Learning\\
2001 Third International Symposium on Adaptive Systems \\
}


\section{Professional Service}{Reviewer}
        {\\
          2021 IEEE Transactions on Neural Networks and Learning Systems\\
          2014 Textbook Reviewer for Elsevier\\
          2013 IEEE Transactions on Neural Networks and Learning Systems\\
          2013 Chemical Engineering Science\\
          2013--2014 Journal of Advances in Robotics and Automation (served on editorial board)\\
2012 International Conference on Artificial Intelligence\\
2010 Journal of Machine Learning Research\\
2004 International Conference on Machine Learning\\
2002 IASTED International Conference on Applied Informatics (AI 2003)\\
2001 IEEE Transactions on Pattern Analysis and Machine Intelligence\\
2001 International Symposium on Adaptive Systems\\
2000 IEEE Transactions on Pattern Analysis and Machine Intelligence \\ 
1999 Journal of Experimental and Theoretical Artificial Intelligence\\
1999 IEEE Transactions on Knowledge and Data Engineering\\
\rm
1998 American Journal of Mathematical and Management Sciences\\
}


        \section{}{North Central North America ACM Programming Competition}
        {\\
        2024 Judge/Problem developer\\
        2023 Head Judge/Problem developer\\
        2022 Head Judge/Problem developer\\
        2021 Judge/Problem developer\\
        2020 Head Judge/Problem developer\\
        2019 Judge/Problem developer\\
        2018 Head Judge/Problem developer\\
        }


%\section{}{Bachelor's Students}
%{\\Todd Quasny, graduated in December, 2001
%}

%\section{Languages \& Systems }{}
%        {C, C++, Pascal, Modula-2, Modula-3, PL/1, Lisp,    
%        Prolog, Ada, LISP, 8051 assembly, 80x86 assembly, 
%        VAX Macro assembly, IBM 360 assembly, 680x0 assembly, 
%        6811 assembly, MIPS assembly, Unix, Macintosh, X11, Motif, 
%        Athena, SQL, Perl, HTML, OpenGL, Java. }
%\pagebreak

%\section{References}{}{Available upon request}
%\section{References}{}

%% \section{References}{Adele E. Howe,}{Department of Computer
%%   Science, Colorado State University, Fort Collins, CO 80523.  Telephone:
%%   (970)491--4192, Email: howe@cs.colostate.edu}

%% \section{References}{William M. Marcy,}{Department Chair and Former Provost, Texas Tech
%%   University, Department of Computer Science, Box 43104, Lubbock, TX 79409,
%% Telephone: (806)742--3970, Email: william.marcy@ttu.edu}

%% \section{}{Robin R. Murphy,} {Department of Computer Science and Engineering,
%% 333 H.R.\ Bright Building,
%% Texas A\&M University, 
%% College Station, TX 77843-3112,
%% Telephone: (979)845--2015, 
%% Email: murphy@cs.tamu.edu
%% }

%% \section{}{Donald C. Wunsch II, } {M.K.\ Finley Distinguished Professor of Computer Engineering,
%% Department of Electrical and Computer Engineering, 1870 Miner Circle,
%% 131 Emerson Electric Company Hall,
%% Rolla, MO 65409, Telephone: (573)341--4521, Email:  dwunsch@ece.umr.edu}

%% \section{}{Susan Mengel,}{Associate Chair, Department of Computer Science, 
%%   Texas Tech University, Box 43104, Lubbock, TX 79409. Telephone: (806)
%%   742--3527 Email: susan.mengel@ttu.edu}



%% \section{}{Henry Hexmoor,}{Department of Computer Science,
%% Southern Illinois University at Carbondale,
%% Faner Hall, Room 2130,
%% Carbondale, IL 62901, 
%% Telephone: (618)453--6047,
%% Email: Hexmoor@gmail.com }

%\section{}{Todd M. Quasny,}{(student) Department of Computer Science,
%  Texas Tech University, Box 43104, Lubbock, TX 79409.  Telephone: (806)
%  438--8633 Email: todd@quasny.com}

%\section{}{J. Ross Beveridge,}{Department of Computer Science,
%        Colorado State University, Fort Collins, CO 80523.  Telephone: 
%        (970) 491--5877 Email: ross@cs.colostate.edu}

%\section{}{Darrell Whitley,}{Department of Computer Science,
%        Colorado State University, Fort Collins, CO 80523.  Telephone:
%        (970) 491--5373 Email: whitley@cs.colostate.edu}


%\section{}{W. J. Bryan Oldham,}{Department of Computer Science, Texas Tech
%        University, Box 43103, Lubbock, TX 79409. Telephone: 
%        (806)742--3527 Email: oldham@cs.coe.ttu.edu}

%%        {Professor Thomas M. English   Department of Computer Science, Texas
%%        Tech University, P.O. Box 43104, Lubbock, TX 79409. Telephone
%%        (806)742-3527(w) (806)791-5774(h). Email: TomEnglish@ttu.edu }
%%

%\section{}{Pat DeLaune,}{Texaco Exploration and Production
%        Technology Department. Telephone: (713)954-6030 }

\leftcolwidth{2em}
\colsep{0em}
\indentitems

\renewcommand\descriptionlabel[1]{\hspace\labelsep\hspace{-2em}\normalfont\bfseries#1}

\ifthenelse{\boolean{prospectus}}
{
\newpage
%\setcounter{page}{1}

\name{\begin{minipage}{\textwidth}\vspace*{-\baselineskip}\begin{center}Larry
D. Pyeatt \\ Statement of Teaching Goals \end{center}\end{minipage}}

\address{\myaddress}

\line
\setlength{\parskip}{\baselineskip}
\setlength{\parindent}{1em}
\thispagestyle{empty}
\chead{Larry D. Pyeatt \\ Teaching Goals}

I take teaching very seriously and strive to do the best job that I can.  I
have worked in industry and routinely bring that experience to the
classroom to help prepare students to enter the work-force. Several students
have attributed their success in industry to taking one or more of my
classes.  My overall philosophy of teaching can be described as follows:

\begin{description}

  \item[Continuous Improvement:] Not only is Computer Science a rapidly
  changing field, but new  pedagogies are being developed all of
  the time. If a course does not change, then it becomes outdated. This is
  true of any field, but especially true for Computer Science. Thus, I work
  for continuous improvement in my course materials, content, and teaching
  style.  This applies to not only the classes that I am teaching, but also
  to my duty to help determine and improve the curriculum for the
  department.

  \item[Active Learning:] Active learning gets the student involved so that
  they learn the material at a deeper level than rote memorisation.
  In-class discussions and course projects are excellent ways to involve
  the students.  The homework and projects should be chosen carefully to
  reinforce the most important concepts in the course.  As students
  progress and mature, they should take more of the responsibility for
  learning. At some point, they can become their own teachers.  That
  is the point at which they are truly educated.

  \item[Appropriate Rewards:] Students should get the grade they earn.
  What many students want is to get the highest possible grades for the
  least amount of work. That is natural and should be expected.  However,
  teachers have a responsibility to display fairness and integrity.  It is
  important to set expectations, tell the students what the expectations
  are, and tie grades to how well the students meet those expectations.

\end{description}

Courses that I enjoy teaching include robotics, artificial intelligence,
digital logic, architecture, operating systems, assembly language, and
real-time systems.   Courses that I would like to teach if
given the opportunity include genetic algorithms,
speech recognition, planning, machine vision, 
Markov decision processes, discrete mathematics, data structures, system
administration, and compiler construction.  In addition to these
preferences, I am competent and willing to teach any traditional computer
science course at either the undergraduate or graduate level.


My greatest teaching achievement involves a student who had a low GPA and
was in danger of dropping out of the program. He confided in me that his
dream was to be a mission controller for NASA.  He also indicated that he
was interested in robotics, so I told him that I would work with him on two
conditions: he was to make a 4.0 GPA in the coming semester, and meet with
me weekly for an independent study in reinforcement learning.  At the end of
the semester, he had achieved all A's and had a good basic understanding of
reinforcement learning.  More than that, our relationship had developed into
a mentorship. By Fall of his senior year, he was doing research. He
published his first paper as a senior and published another in his first
year of graduate school. Not only did he blossomed academically, but he also
decided to work towards a PhD.  

About one year into his dissertation, we were working on a research project
with a group from NASA and he got the opportunity of a lifetime.  He was
offered a position as a mission controller on the International Space
Station.  I was sad to see him take the position, but also happy for him.
Not many people get to achieve their dreams. My mentorship of him has given
me a new perspective on teaching and advising: Some students need a teacher
to get them interested and involved, and I can be that teacher.  Nothing
could be more personally rewarding.





%%% Local Variables: 
%%% mode: latex
%%% TeX-master: "career_goals"
%%% End: 

\newpage
%\setcounter{page}{1}

\name{\begin{minipage}{\textwidth}\vspace*{-\baselineskip}\begin{center}Larry
D. Pyeatt \\ Statement of Research Goals \end{center}\end{minipage}}

\address{\myaddress}

\line
\setlength{\parskip}{\baselineskip}
\setlength{\parindent}{1em}
\thispagestyle{empty}
\chead{Larry D. Pyeatt \\ Research Goals}


My main research interests are probabilistic AI techniques and mobile
robot navigation in complex unstructured environments.  My most recent
work deals with robust control and learning for partially observable,
uncertain, and non-stationary environments, and using machine learning
techniques to control sedation of surgical patients.  My research
interests can be divided into three general areas:
\begin{description}
\item[Probabilistic Techniques:] I am investigating partially
  observable Markov decision process (POMDP) techniques.  My research
  in this area is aimed at efficiently finding exact and approximate
  policies for POMDP problems.  So far, the application domain for
  this work has been autonomous mapping and navigation of indoor
  environments.  My next goal is to extend the current techniques to
  work in large outdoor environments. Outdoor robotics pose several
  grand challenges in the area of mobile robotics.

\item[Sensor Modeling:]
 In order to use the probabilistic mapping techniques, it is necessary to
 convert a stream of sensor data into a stream of local maps.  Data from
 multiple local maps, possibly generated from different sensors, can be
 fused to form a global map.  For some sensors, such as laser and sonar,
 the sensor model is well understood and easy to implement.  However, no
 good model for stereo vision exists.

\item[Learning Actions:] The behaviors that humans perform are quite
  often either completely reflexive or were learned at an early age
  and have since become reflexive in nature. Higher level behaviors
  are ordered sets of these sub-cognitive behaviors. I am very
  interested in developing solutions for learning these low-level
  sub-cognitive behaviors in order to provide them to intelligent
  agents for ordering in high-level behaviors. For my dissertation
  work, I developed a framework for using POMDP based navigation with
  reinforcement learning (RL) to provide adaptive low-level actions.
  This work was a proof of concept and was done completely in
  simulation.  Some of my current work is aimed at extending the
  architecture to run on a real robot.
\end{description}

My future research plans involve continued effort in learning and
control, and application of probabilistic techniques in domains other
than robotics.  In particular, I am interested in applying
probabilistic machine learning techniques in healthcare and clinical
settings. I have also begun investigating distributed computing and
adaptive wireless networks to support computation and communication
between computers, sensors, and robots.  I would also enjoy working on
issues of human-robot interaction, including gesture recognition,
learning through imitation, and understanding high-level spoken
commands.


}{}

\end{document}

